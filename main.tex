\documentclass[a4paper]{memoir}
\usepackage[utf8]{inputenc}
\usepackage[swedish]{babel}
\usepackage{gb4e}
\usepackage{enumitem}

\newcommand{\eline}{\\[\baselineskip]} % empty line

\newenvironment{mytable}[2]
{
  \eline
  \begin{tabular}{#1}
  \hline #2\\ \hline
}
{ \end{tabular} }
\newcommand{\myheader}[1]{ \hline #1\\ \hline }
\newenvironment{wlistvisv} % word list, vikamål-svenska
%{ \begin{\mytable} \myheader{Vikamål & Svenska} }
{ \begin{mytable}{ l l }{Vikamål & Svenska} }
{ \end{mytable} }
\newenvironment{wlistsipl} % word list, singular-plural
{ \begin{mytable}{ l l }{Singular & Plural} }
{ \end{mytable} }
\newenvironment{wlistmnf} % word list, mask-neutr-fem
{ \begin{mytable}{ l l l }{Maskulinum & Neutrum & Femininum} }
{ \end{mytable} }
\newenvironment{wlistvcon} % word list, verb conjugation
{ \begin{mytable}{ l l l l l l l }
  {
%    \multicolumn{3}{c}{Singular} & \multicolumn{3}{c}{Plural}\\
%    \hline
    1 & 2 & 3 & 1 & 2 & 3 & Svenska
  }
}
{ \end{mytable} }
\newenvironment{wlistinfl} % word list, inflection of adjectives
{ \begin{mytable}{ l l l }{Positiv & Komparativ & Superlativ} }
{ \end{mytable} }
\newenvironment{rlist} % rule list
{ \begin{enumerate}[label=\Alph*] }
{ \end{enumerate} }

\begin{document}

\title{Vikamål}
\author{Joel Sjögren}
\date{2013-12-31}
\maketitle
\tableofcontents

%\part
%\chapter
%\section
%\subsection

\part{Substantiv}
  \chapter{Genus}
    Vikamålet har tre genus: \emph{maskulinum}, \emph{neutrum} och \emph{femininum}.\\Svenskan däremot klumpar ihop maskulinum och femininum.
    \section{Bestämd form}
      I vikamål visas genus med den bestämda artikeln. Den används bara vid betoning (\emph{den} mannen) alltså som i svenskan.
      \begin{wlistmnf}
        dånda Karrn & eda Bordä & doda Lampu\\
        dånda Rackan & eda Govä & doda Männischu\\
        . & . & doda Nåli
      \end{wlistmnf}
    \section{Obestämd form}
      .
      \begin{wlistmnf}
        jenn Karr & jett Bord & je Lampa\\
        jenn Rackä & jett Gov & je Männisch\\
        . & . & je Nål
      \end{wlistmnf}
    \section{Maskuliner}
      \begin{rlist}
        \item Ord som betecknar män.
        \begin{wlistvisv}
          jenn Karr & en man\\
          jenn Tupp & en tupp\\
        \end{wlistvisv}
        \item Ord som i plural på svenska slutar \emph{-ar}.
        \begin{wlistvisv}
          jenn Dag & en dag\\
          jenn Stol & en stol\\
          jenn Måne & en måne\\
        \end{wlistvisv}
        \item Ord som beskriver en sak utifrån dess uppgift. Dessa känns igen genom att de slutar på \emph{-er} (motsvarande det svenska \emph{-are}) eller \emph{-or}.
        \begin{wlistvisv}
          jenn Dator & en dator (\emph{lat. givare})\\
          jenn Motor & en motor\\
          jenn CD-Spiler & en CD-spelare\\
        \end{wlistvisv}
        \item Ord som slutar på \emph{-tion}.
        \begin{wlistvisv}
          jenn Station & en station\\
        \end{wlistvisv}
        \item Ord som slutar på \emph{-at}.
        \begin{wlistvisv}
          jenn Tomat & en tomat\\
          jenn Kamrat & en kamrat\\
        \end{wlistvisv}
      \end{rlist}
    \section{Neutrer}
      \begin{rlist}
        \item Ord som liknar svenska ord och i svenskan har den obestämda artikeln \emph{ett}.
        \begin{wlistvisv}
          jett Trejd & ett träd\\
          jett Öjs & ett Hus\\
          jett Snöre & ett snöre\\
        \end{wlistvisv}
      \end{rlist}
    \section{Femininer}
      \begin{rlist}
        \item Ord som slutar på \emph{-a}.
        \begin{wlistvisv}
          je Sejda & en sida\\
          je Tröja & en tröja\\
          je Öna & en höna\\
        \end{wlistvisv}
        \item Ord som slutar på \emph{-ing}. (Notera att både \emph{ng}- och \emph{g}-ljudet uttalas.)
        \begin{wlistvisv}
          je Bogsering & en bogsering\\
          je Inspielning & en inspelning\\
        \end{wlistvisv}
      \end{rlist}
  \chapter{Plural}
    Vikamålet har samma numerus som svenskan: \emph{singular} och \emph{plural}.\eline
    I alla genus plural heter den bestämda artikeln \emph{demda}. Den obestämda artikeln flera heter \emph{fler}.\eline
    Ombildning från singular till plural är i många fall enklare än i svenskan. Istället för att välja bland \emph{-ar} och \emph{-er} använder man ofta bara \emph{-er} (som uttalas lite mer åt \emph{är}-hållet).
    \begin{wlistsipl}
      je Bok & fler Böker \\
      jenn Stol & fler Stoler \\
    \end{wlistsipl}\eline
    Som synes ovan liknar omljud, det vill säga byte av vokal, svenskan.\eline
    Nedan följer några fall där pluralbildningen är svårare.
    \begin{rlist}
      \item Ord som i singular redan slutar på \emph{-er} får ytterligare ett \emph{-er}.
      \begin{wlistsipl}
        jenn CD-Spiler & fler CD-Spilerer\\
      \end{wlistsipl}
      \item Plural av ord som slutar på \emph{-a} bildas genom utbyte mot ändelsen \emph{-ur} (uttalas \emph{urr}).
      \begin{wlistsipl}
        je Tröja & fler Tröjur\\
        je Sejda & fler Sejdur\\
      \end{wlistsipl}
      \item Förkortningar böjs med samma svårighet som i svenskan.
      \begin{wlistsipl}
        jenn TV & fler TV-apparater
      \end{wlistsipl}
      \item Neutrala ord heter samma sak i singular som i plural. Undantaget är \emph{jett trejd, fler trejn}. [huvud?]
      \begin{wlistsipl}
        jett Bord & fler Bord\\
        jett Snöre & fler Snöre\\
        jett Öjs & fler Öjs\\
      \end{wlistsipl}
    \end{rlist}
%  \chapter{Kasusformer}
    \chapter{Verb}
      \section{Presens}
        .
        \begin{wlistvcon}
          itter & itter & itter & ittum & ittir & itta & hittar\\
          a & a & a & am & avir & a & har\\
          e & e & e & em & evir & e & är\\
        \end{wlistvcon}
      \section{Imperfekt}
        .
        \begin{wlistvcon}
          itted & itted & itted & ittedum & ittedir & itted & hittade\\
          ad & ad & ad & adum & adir & ad & hade\\
          va & va & va & vam & vair & va & var\\
        \end{wlistvcon}
      \section{Perfekt}
        .
        \begin{wlistvisv}
          itta & hittat\\
          aft & haft\\
          ve & varit\\
        \end{wlistvisv}
      \section{Imperativ}
        Exempel andra person plural (?):
        \begin{wlistvisv}
          Kommir jon! & Kom hit /med er/!\\
        \end{wlistvisv}
    \chapter{Adjektiv}
      \section{Komparation}
        .
        \begin{wlistinfl}
          snabb & snabber & snabbest\\
          rod & roder & rodest\\
          stur & störrer & störst\\
          liten & minder & minst\\
        \end{wlistinfl}
      \section{Böjning (inflektion)}
        Adjektiv böjs inte: \emph{jenn snabb bil, dånda snabb biln}.
      \section{Bildning av adverb}

\end{document}








